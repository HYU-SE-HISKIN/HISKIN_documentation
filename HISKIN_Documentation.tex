\documentclass[conference]{IEEEtran}
\IEEEoverridecommandlockouts
\usepackage{cite}
\usepackage{amsmath,amssymb,amsfonts}
\usepackage{algorithmic}
\usepackage{graphicx}
\usepackage{textcomp}
\usepackage{graphicx}
\usepackage{indentfirst}
\usepackage{caption}
\usepackage{tabularx}
\hbadness=9
\vbadness=9
\hfuzz=20pt
\def\BibTeX{{\rm B\kern-.05em{\sc i\kern-.025em b}\kern-.08em
    T\kern-.1667em\lower.7ex\hbox{E}\kern-.125emX}}
\begin{document}

\title{HI-SKIN\\

}

\author{\IEEEauthorblockN{1\textsuperscript{st} CHAN MIN KIM}
\IEEEauthorblockA{\textit{Dept. Information System} \\
\textit{College of Engineering}\\
\textit{Hanyang University}\\
Seoul, Korea \\
han2cmk@hanyang.ac.kr}
\and
\IEEEauthorblockN{2\textsuperscript{nd} SEOK YOUNG KIM}
\IEEEauthorblockA{\textit{Dept. Information System} \\
\textit{College of Engineering}\\
\textit{Hanyang University}\\
Seoul, Korea \\
kim2653seok@gmail.com}
\and
\IEEEauthorblockN{3\textsuperscript{rd} HAE RYUNG CHA}
\IEEEauthorblockA{\textit{Dept. Information System} \\
\textit{College of Engineering}\\
\textit{Hanyang University}\\
Seoul, Korea \\
haeryung8@hanyang.ac.kr}
\and
\IEEEauthorblockN{4\textsuperscript{th} YU JIN PARK}
\IEEEauthorblockA{\textit{Dept. Information System} \\
\textit{College of Engineering}\\
\textit{Hanyang University}\\
Seoul, Korea \\
jinee22@hanyang.ac.kr}

}

\maketitle
\thispagestyle{plain}
\pagestyle{plain}

\begin{abstract}

Nowadays, with the increasing importance of skincare, a growing number of busy modern people are trying skincare at home by purchasing beauty devices instead of going to a dermatologist. For these customers, companies such as LG and FOREO are providing beauty devices and applications that can be linked to them. However, managing at home with a beauty device has several weaknesses in place of dermatology. First, there is a lack of communication. Many companies are trying to provide communication such as management methods, but it is still not enough to replace dermatology. The second is that it ends with care. Skin care requires care tailored to one's own skin type even after care, but there is a lack of action. To develop an application that can compensate for these weaknesses, our team is trying to develop a user-friendly application called HI-SKIN that can work with beauty devices for those who have difficulty taking care of their skin. It is managed by a beauty device registered by a user, and at the same time, it enables communication with users by using AI voice recognition technology. It provides communication functions on various topics such as skin care tips, daily conversations, and management methods according to the user's mood in addition to the existing function of providing guidance on how to use devices. Through these functions, we expect users can enjoy the same service as meeting a doctor in person. And also, through additional challenges program, users can take care of their skin not only at the moment they use the device, but also after care or normally. Depending on the user's skin type, the challenge provides a user-friendly skin care function while allowing the user to perform missions such as moisture soothing pack and drinking more than 1L of water per day. In these days, when the desire to be beautiful continues to increase, concerns about good skin are inevitably growing, and our application HI-SKIN will be a good solution in this situation. \cite{zhang2020impact}
\end{abstract}

\begin{IEEEkeywords}
skin care, HI-SKIN, User-Friendly Application, AI, Beauty Device, Communication
\end{IEEEkeywords}

\begin{table} [h]
    \caption{Task Distributions for Each Member}
    \centering
    \begin{tabular}{l|l|l}
    \hline
    \textit{\textbf{Roles}} & \textit{\textbf{Name}} & \textit{\textbf{Description}}
    & & & \\ 
    \hline
   \textit{\textbf{\begin{tabular}[c]{@{}l@{}}Software \\ Developer\\(Front-End)\end{tabular}}} & \textit{\textbf{\begin{tabular}[c]{@{}l@{}}HAE \\ RYUNG \\ CHA\end{tabular}}}& \begin{tabular}[c]{@{}l@{}}A software developer(Front-End) designs\\ applications using languages such as\\ HTML, CSS and React-Native. Key\\ responsibilities include developing responsive\\ interfaces, implementing interactive\\ features and navigation components, and \\optimizing code for performance. Essential\\ skills and qualifications include \\knowledge of responsive web design, \\mobile-first development, problem-solving abilities,\\ and excellent communication skills.\end{tabular} \\ \hline
   \textit{\textbf{\begin{tabular}[c]{@{}l@{}}Software\\ Developer\\(Back-End)\end{tabular}}} & \textit{\textbf{\begin{tabular}[c]{@{}l@{}}SEOK \\ YOUNG \\ KIM\end{tabular}}}& \begin{tabular}[c]{@{}l@{}}Backend is a technology\\ that manages servers or databases, \\areas that users in web applications \\do not see. The backend is responsible \\for managing data or running servers \\so that users can provide the information \\they want. In other words, the backend is\\ about dealing with what users at the \\front-end want to-do. As a result, backend \\developers engage in various development \\activities, including system component \\work, API creation, library creation, and \\database integration. \end{tabular} \\ \hline
   \textit{\textbf{\begin{tabular}[c]{@{}l@{}}Software\\ Developer\\(Machine\\Learning)\end{tabular}}} &\textit{\textbf{\begin{tabular}[c]{@{}l@{}}YU \\ JIN \\ PARK\end{tabular}}} & \begin{tabular}[c]{@{}l@{}} A Machine Learning Engineer is\\ responsible for designing and developing\\ machine learning systems, implementing \\appropriate ML algorithms, conducting\\ experiments, and staying updated with \\the latest developments in the field. \\They work with data to create models\\, perform statistical analysis, and train\\and retain systems to optimize performance.\\ Their goal is to build efficient self-learning\\applications and contribute to \\advancements in artificial intelligence. \end{tabular} \\ \hline
   \textit{\textbf{\begin{tabular}[c]{@{}l@{}}Project\\ Designer\\(Documentation)\end{tabular}}} & \textit{\textbf{\begin{tabular}[c]{@{}l@{}}CHAN \\ MIN \\ KIM\end{tabular}}}& \begin{tabular}[c]{@{}l@{}} A project designer's primary role is\\ to craft a product with a user-centric\\ approach, demonstrating a deep capacity \\to empathize with and comprehend\\ the user's journey. This pivotal role \\involves shaping the fundamental structure\\ of a product or service. The project designer\\ is entrusted with fostering seamless \\communication among team members \\and fostering collaboration. The project \\designer's central focus lies in enhancing\\ the product or service's usability,\\ adapting the product's design as \\necessary to achieve superior outcomes.
 \end{tabular} \\ \hline
   \end{tabular}
\end{table}

\clearpage

\section{Introduction}
\subsection{Motivation} 
According to a survey conducted by market research firm Embrain Trend Monitor in 2022, significant changes in awareness about skincare have been observed. The survey targeted 1,000 adults aged 19 to 59 and examined their perceptions related to skincare. The results showed a decrease in satisfaction with their skin condition from 41.4\% in 2019 to 37.8\%. Furthermore, 64.2\% of respondents reported increased concerns about their skin. This data suggests that many people have become less satisfied with their skin condition recently, leading them to desire healthier and more vibrant skin.
\\Several factors contribute to this shift in perception. Firstly, in today's era of social media, platforms like Instagram, YouTube, and Facebook provide easy access to information about skincare and beauty. Influencers and beauty experts share product reviews, skincare tips, and personal experiences, inspiring users to take a greater interest in skincare and develop their skincare routines.
\\Secondly, there is a rising interest in health and well-being. Healthy skin is perceived as a crucial indicator of overall health and well-being, prompting many individuals to pursue better health through skincare. Skincare is not only seen as a means to enhance one's appearance but also as a way to improve skin health.
\\As a result of these trends, interest in skincare and beauty devices has been on the rise. According to Embrain Trend Monitor's survey, there is a growing interest in home beauty and skincare. 81.2\% of women perceive skincare devices as popular, and products like those from LG Pra.L and Medique have been predicted to become more prevalent. These devices are considered cost-effective alternatives that allow for convenient skincare at home compared to visiting dermatologists.
\begin{figure} [h]
    \centering
    \includegraphics[width=.6\columnwidth]{fig/PraL1.jpg}
    \caption{LG Pra.L Care}
\end{figure}
\\However, this increased interest and device usage come with some skepticism and concerns. Users have doubts about the safety and effectiveness of skincare devices, and there are worries about the potential side effects of performing skincare independently. In fact, a significant percentage of consumers have experienced side effects when using home beauty devices. According to a 2019 survey by the Consumer Education Center, 10\% of users reported experiencing side effects. This is often due to a lack of information about potential side effects associated with these devices, highlighting a communication gap between users and skincare devices that differs from traditional dermatology practices.
\\In conclusion, the changing perceptions of skincare and home beauty are the result of various interacting factors, leading to emerging market trends and consumer demands related to skincare devices. Therefore, we have decided to create a more comfortable, safe, and user-friendly app to accompany skincare devices to address these evolving consumer needs. \cite{kim2022skincare}


\subsection{Problem Statement}\label{SCM}
\begin{itemize}
\item [1] Many people are currently experiencing significant difficulties and discomfort while using skincare beauty devices. The reason for this is that skincare beauty devices often fail to provide users with sufficient information about the potential risks and side effects, unlike procedures conducted by professional skincare experts or dermatologists.\\ \cite{Song2022beautydevice}
\item [2] Consumers often struggle to establish consistent skincare routines even when using skincare beauty devices. This is primarily because these devices typically provide instructions and information specific to the device itself, often lacking personalized advice based on individual users' skin conditions and needs.\\
\item [3] Consumers consider homecare devices as alternatives to dermatological services, seeking a similar experience even if not entirely equivalent. However, currently available skincare beauty devices often fall short in providing users with comprehensive information compared to dermatologists, making users perceive them as somewhat "clinical." Furthermore, when receiving skincare services at a dermatologist's office, the interaction with the dermatologist adds an element of "fun" or engagement to the experience, which is lacking when using skincare beauty devices. Consequently, many consumers desire to acquire enjoyable or informative aspects akin to visiting a dermatologist while using skincare beauty devices.
\end{itemize}
\\
\\

\subsection{Research on Related Software}\label{SCM}
\begin{enumerate}
    \item[A.] LG Pra.L Care
    \begin{figure} [h]
    \includegraphics[width=\columnwidth]{fig/PraL2.png}
    \caption{LG Pra.L Care}
    \end{figure}
\item[] LG Pra.L Care is an app created by the South Korean conglomerate LG, designed to enhance the effectiveness of skincare routines for users who own LG Pra.L products. This app offers several key features, including the ability to determine the user's skin type through a 15-question survey and provide recommendations on how to use LG Pra.L products more effectively. Additionally, it offers daily skincare tips based on the weather and air quality and provides a ranking system for skincare products tailored to the user's skin type, assessing their suitability as a percentage match.
\\
\\


\item[B.] HWAHAE
\begin{figure} [h]
    \includegraphics[width=\columnwidth]{fig/hwahae1.png}
    \caption{HWAHAE}
\end{figure}
\item[] HWAHAE is the leading domestic cosmetics app in South Korea, ideal for consumers who are unsure about which skincare or cosmetic products to purchase. This app's primary features include ingredient analysis and user reviews. The review feature mandates users to list both the pros and cons of the products, allowing consumers to gain in-depth insights into the products they intend to buy. Users can also explore popular products by category through the ranking feature, identifying which cosmetics are currently trending. Furthermore, the "HWAHAE PLUS" section offers beauty-related information, including details about cosmetics and skincare.
\\

\\
\item[C. ] Glowpick
\begin{figure} [h]
    \centering
    \includegraphics[width=.5\columnwidth]{fig/glowpick1.jpg}
    \caption{Glowpick}
\end{figure}
\item[] "Glowpick" is an app that lives up to its slogan of "Finding good cosmetics is a good habit" by offering users cosmetic rankings based on honest product reviews from real consumers. This app not only includes products launched domestically but also registers and sells products from various sources, including roadshops, drugstores, department stores, and even products not officially released in South Korea.\\Furthermore, Glowpick provides category-specific rankings for products available in each offline purchasing channel, such as Olive Young, Watsons, LOHBs, and Aritaum. It also offers a wealth of information, including diverse sale details, makeup buying tips, and a review search feature that helps consumers discover great cosmetics through user reviews. Additionally, it provides comprehensive information about the ingredients contained in each cosmetic product.
\clearpage

\item[D. ] FOREO For You
\begin{figure} [h]
    \centering
    \includegraphics[width=.5\columnwidth]{fig/FOREO1.png}
    
    \caption{FOREO For You}
\end{figure}
\item[] FOREO For You is an app that works in conjunction with FOREO's skincare devices, offering skin analysis and personalized skincare guidance. Additionally, this app monitors skin conditions and assists users in using the devices effectively.
\\
\\

\item[E. ] Clarisonic Mia Smart
\begin{figure} [h]
    \includegraphics[width=\columnwidth]{fig/Clarisonic1.jpg}
    \caption{Clarisonic Mia Smart}
\end{figure}
\\
\item[] Clarisonic Mia Smart is an application that connects to Clarisonic's skincare devices, providing skin analysis and customized skincare routines to users through videos and photos demonstrating how to use the skincare device effectively.

\end{enumerate}
\\

\section{REQUIREMENT ANALYSIS}

\subsection{Create an Account}
If users are new to the app, they can create an account by clicking the "Sign Up" button. After clicking the button, users will be asked to answer a few questions to provide their information and create the account. The following items are the information that is needed for creating an account:
\begin{itemize}
    \item Name
    \item Gender
    \item Nickname
    \item ID 
    \item Password
\end{itemize}
After entering all the information, user will be taken to the login page. In addition, user can edit their information on "My Page".\\
\subsection{Login}
Signing in is a crucial step for all members to gain entry to the application, acting as the pathway to harness the complete spectrum of features offered by HI-SKIN. The process of logging in involves inputting the ID and password that were initially provided during the registration phase. This mandatory login procedure constitutes a fundamental element of the HI-SKIN platform, granting users access to a tailored skincare experience and empowering them to proficiently oversee their skin. The authentication mechanism, employing an ID and password, plays a pivotal role in upholding account security and ensuring the protection of confidential information.\\
\subsection{Skin Type Test}
The app offers a "Skin Type Diagnosis" feature that helps users determine their skin type. Users answer a series of questions about their skin through a survey format. The app then automatically provides information about the user's skin type based on their responses.\\
\subsection{Cosmetic Recommendations}
Users can receive recommendations for cosmetics that suit their skin types. The app searches for keywords based on the user's skin type on the "Olive Young" website and provides information on the top-ranked product in terms of popularity.\\
\subsection{Device Registration}
Users can register their skincare devices within the app by clicking the "Device Registration" button. \\
\subsection{Interactive Voice AI Communication}
The app enables two-way communication with skincare devices via voice recognition. This interaction allows for daily conversations, addressing concerns, and providing personalized skincare product recommendations based on user preferences.\\
\subsection{Community}
The app offers a community feature, allowing users to communicate with each other. Users can share their skincare concerns and effective skincare tips through the community, making interaction easier with comments and like buttons. \\
\subsection{Skincare Challenges}
Users participate in a skincare challenge and receive challenge scores. The challenge categories include important skincare routines selected by the Korean Dermatological Association, such as sun protection, hydration, skincare/cleansing, and avoiding unhealthy dietary habits for skin care. Challenge scores are recorded in the user database. \\
\subsection{Skin Report}
Users can visualize their challenge scores for the past week on a graph. Additionally, they have the option to review the average challenge scores over 3 months and 6 months. This allows users to assess the status of their skincare management, providing valuable insights for establishing and refining their skincare routines.\\
\subsection{My Page}
Users can access various features on their My Page, including updating user registration information, retaking the skin type test, and reviewing their skin reports.\\

\section{development environment}
\subsection{Choices of Software Development Platform}
\begin{enumerate}
    \item[a.] Development Platform
    
    \begin{enumerate}
    \item[1.] Windows \cite{novac2017comparative} \cite{stallings2005windows}
    \item[] Windows operating system is popular among both users and developers for various reasons. For users, it provides a familiar and user-friendly interface, making everyday computing tasks straightforward. Additionally, a wide range of software and games are predominantly supported and optimized for Windows, allowing users to access diverse applications seamlessly.
    \\
    Developers benefit from Windows through its rich development tools and integrated development environments. Robust tools like Visual Studio support various programming languages and frameworks, facilitating apps and web development and enterprise solution building. Windows provides an environment for developing applications for different platforms and is optimized for game development and graphic work.
    \\
    Moreover, Windows exhibits excellent compatibility with various hardware and software, enabling developers to work across different environments more efficiently. This feature is particularly crucial in business and enterprise environments, where many companies adopt Windows to develop and utilize enterprise-level software. Therefore, Windows is acknowledged as a powerful operating system that caters to a broad spectrum of tasks and requirements for both users and developers. \\

    \item[2.] MacOS \cite{sherry2013foundation}
    \item[] MacOS is a favored operating system among web and app developers for several reasons. Its Unix-based foundation provides a powerful command-line interface, making it conducive for development tasks. The terminal offers a robust environment for running scripts, installing packages, and executing various developer tools, enhancing efficiency in development workflows.
    \\
    Developers appreciate MacOS for its compatibility with a wide array of programming languages and frameworks. Xcode, the integrated development environment exclusive to MacOS, stands out for creating applications across Apple's ecosystem, including MacOS, iOS, WatchOS, and TvOS. The development environment, combined with the availability of software development kits (SDKs), facilitates the creation of high-quality, native applications.
    \\
    Moreover, MacOS is highly regarded for its stability and security, essential factors for developers handling sensitive data and applications. The system's stability ensures a reliable platform for coding and testing, while its security features offer a protective environment for sensitive development projects.
    \\
    For web developers, MacOS supports a variety of web development tools, including popular editors like Visual Studio Code, Sublime Text, and Atom. The operating system's compatibility with web technologies, such as HTML, CSS, and JavaScript, along with its Unix core, provides an ideal environment for web development projects.
    \\
    Lastly, the integration of hardware and software in Apple products often enhances the development experience. The seamless connection between Apple devices and the ability to test applications on various Apple products contributes to the appeal of MacOS among app developers aiming to create high-quality, well-integrated software for Apple users. Overall, MacOS is a preferred operating system for web and app developers due to its strong development tools, Unix-based environment, and seamless integration with Apple's hardware and software ecosystem. \\

    \item[3.] Android \cite{singh2014overview}
    \item[]The Android operating system stands as one of the most prevalent mobile OS platforms in use today. Rooted in the Linux kernel, it is the brainchild of Google, tailored primarily for smartphones and tablets. Its open-source architecture has spurred unprecedented growth, making it the swiftly burgeoning choice among users and developers. This open nature allows for easy customization and integration of advanced functionalities, aligning with the dynamic needs of mobile technology.\\
    With over 1.5 billion applications and games downloaded monthly from the Google Play Store, the Android OS is revered for its robust development framework. Users and software developers leverage its power to create a diverse array of applications for a broad spectrum of devices. To support seamless software development, Android furnishes the Android Software Development Kit (SDK), employing the Java programming language. This comprehensive kit comprises a debugger, libraries, a handset emulator using QEMU (Quick Emulator), comprehensive documentation, sample code, and tutorials.\\

    \item[4.] iOS \cite{wukkadada2015mobile}
    \item[]iOS, as the exclusive operating system developed by Apple for its suite of mobile devices, including the iPhone, iPad, and iPod Touch, is renowned for its developer-friendly environment and end-user experience. From a development perspective, the iOS ecosystem offers a robust framework, utilizing languages like Swift and Objective-C within the iOS Software Development Kit (SDK) to create innovative applications.\\
    The closed ecosystem of iOS ensures a more controlled environment for developers, with stringent app submission guidelines and a comprehensive review process, contributing to a high standard of app quality and security within the App Store. The integration of hardware and software is seamless, enabling developers to leverage the full potential of Apple devices and services like Apple Watch, iCloud, and Mac computers, thus providing a unified experience across the Apple ecosystem.\\
    Regular updates to the iOS platform not only introduce new features and enhancements but also address security concerns and bugs, ensuring a stable and secure environment for both developers and end-users. The platform's optimization with the hardware allows developers to create apps that run smoothly and efficiently, enhancing performance and battery life.\\
    Additionally, the incorporation of Siri, Apple's virtual assistant, allows developers to integrate voice commands and other functionalities into their applications, offering a more interactive user experience. The commitment to user privacy and data security through features such as Touch ID and Face ID underscores the priority Apple places on safeguarding user information.\\
    Overall, the iOS environment provides developers with a combination of reliable tools, a secure and controlled ecosystem, and a high-quality user experience, contributing to its appeal among both developers and end-users.\\
    \end{enumerate}

  \item[b.]Tools and Language

    \begin{enumerate}
        \item[1.]Java
        \item[]Java is an object-oriented programming language developed by James Gosling of Sun Microsystems and other researchers. It is one of the most commonly used languages in the web application field and is also widely used in software development for mobile devices including Android. Millions of Java applications are in use today as a result of their popularity among developers for over 20 years. It's a fast, secure and reliable programming language for coding everything from mobile apps and enterprise software to big data applications and server-side technologies. In addition, it is a very suitable language for game development, database processing, big data, and distributed processing. Our team will develop an application, so we will build a backend server using Java.\\

        \item[2.]Spring Boot 
        \item[]The Java Spring Framework is a popular enterprise-class open-source framework for creating production-class standalone applications running on Java Virtual Machine (JVM). Java Spring Boot is a tool that helps you develop web applications and microservices faster and more easily using the Spring Framework through three core functions. Automatic configuration, a self-righteous approach to configuration, and the ability to create standalone applications. These features work together to provide tools to help you set up Spring-based applications with minimal configuration and settings. Because of these advantages, our team decided to use the Spring Boot framework to build backend servers in this project.\\ 

        \item[3.]Hibernate \cite{fisher2010spring}
        \item[]Hibernate is an ORM framework intended to translate between relational databases and the realm of object-oriented development. Hibernate provides a querying interface, using Hibernate Query Language (HQL) or the Hibernate Criteria API. Using hibernate increases productivity because queries can be performed only by method calls without using SQL directly, and it is also excellent in terms of maintenance because it performs the parameters, results, etc. of the DAO related to the table when the table column is changed.Together, Spring and Hibernate are a dynamic duo, capable of simplifying dependency collaboration, reducing coupling, and providing abstractions over persistence operations. Since our team decided to use spring boot for the backend server, we decided to use hibernate, which is well matched with spring boot, to build the database.\\

        \item[4.]Jupyter Notebook \cite{jupyterNotebook}
        \item[]Jupyter Notebook is the most widely-used system for interactive literate programming. It was designed to make data analysis easier to document, share, and reproduce. Jupyter originated from IPython and, in addition to Python, it supports a variety of programming languages, such as Julia, R, JavaScript, and C. It also allows the interleaving of not only code and text, but also different kinds of rich media, including image, video, and even interactive widgets combining HTML and JavaScript.\\

        \item[5.]Google Colaboratory \cite{8485684}
        \item[]Google Colaboratory (also known as Colab) is a cloud service based on Jupyter Notebooks for disseminating machine learning education and research. It provides a runtime fully configured for deep learning and free-of-charge access to a robust GPU. Thus, it can be effectively exploited to accelerate not only deep learning but also other classes of GPU-centric applications.\\

        \item[6.]TensorFlow \cite{Tensorflow}
        \item[]TensorFlow is a machine learning system that operates at large scale and in heterogeneous environments. Its computational model is based on dataflow graphs with mutable state. Graph nodes may be mapped to different machines in a cluster, and within each machine to CPUs, GPUs, and other devices. TensorFlow supports a variety of applications, but it particularly targets training and inference with deep neural networks. It serves as a platform for research and for deploying machine learning systems across many areas. \\

        \item[7.]PyTorch
        \item[]PyTorch is a machine learning library that shows that these two goals are in fact compatible: it was designed from first principles to support an imperative and Pythonic programming style that supports code as a model, makes debugging easy and is consistent with other popular scientific computing libraries, while remaining efficient and supporting hardware accelerators such as GPUs. Our goal with PyTorch is to build a flexible framework to express deep learning algorithms. \\

        \item[8.]Docker
        \item[]Docker was designed in order to simplify the creation, deployment and execution of applications using containers. Containerization allows the user to run applications in a virtual environment by packaging all necessary elements such as files, libraries and other essential components together. Furthermore, containers play a vital role in DevOps processes as an integral part of automated software builds or as part of continuous deployment. \\

        \item[9.]Git \cite{velog-git}
        \item[]Git serves as an effective tool for managing versions, facilitating the seamless integration of changes and updates. However, prior to delving into Git, it's important to grasp the concept of a 'version control system.' Essentially, a version control system captures and monitors modifications made to a file, enabling easy retrieval of any previous iteration when needed. While working on a document, multiple revisions take place, progressing from the initial draft to the ultimate version. Often, files are renamed as 'final,' 'final copy,' 'finalized,' and so on, leading to the replacement of previous versions. This practice can complicate the process of reverting to a specific point in time to understand the alterations made. However, a version control system resolves this challenge. It enables the management of numerous iterations of the same data, facilitating the tracking of changes over time and attributing them to specific contributors. This simplifies the process of reverting to previous or original versions and promptly identifying the responsible individuals for any issues that may arise.\\

        \item[10.]JavaScript \cite{jensen2009type}
        \item[]JavaScript is the main scripting language for Web browsers, and it is essential to modern Web applications. We present a static program analysis infrastructure that can infer detailed information for JavaScript programs using abstract interpretation. The analysis is designed to support the full language as defined in the ECMAScript standard, including its peculiar object model and all built-in functions. The analysis results can be used to detect common programming errors – or rather, prove their absence, and for producing type information for program comprehension.\\

        \item[11.]React Native \cite{waren2016cross}
        \item[]React Native is an open source JavaScript framework for building mobile applications for both iOS and Android devices. It was open-sourced on March 2015 by Facebook, and it's based on the React framework published a few years earlier. \\
        
        \item[12.]Python
        \item[]Python is a high-level, versatile programming language known for its readability and simplicity, making it an excellent choice for beginners and professionals alike. It supports multiple programming paradigms and offers a vast array of libraries and frameworks, enabling developers to create diverse applications, from web development and data analysis to artificial intelligence and scientific computing. Its clear and concise syntax promotes easy comprehension, fostering rapid development and fostering a robust community that contributes to its continuous evolution and widespread adoption.\\

        \item[13.]Bert \cite{affi2021blc}
        \item[]The full name of BERT is Bidirectional Encoder Representations from Transformers, which is a language model representation based on self-attention blocks. The main innovation of the model is the pre-training method, which is trained with Masked LM and next sentence prediction to capture the word and sentence level representation respectively. BERT is pre-trained in different language model tasks using existing unmarked corpora. The pre-trained deep bidirectional model with one output layer can reach state-of-the-art performance in many tasks such as question answering and multi-genre natural language inference. The idea is to have a common architecture that fits many tasks and a pre-trained model that reduces the need for labeled data. for a given token, its vector representation is built by summing the corresponding, word, sub-word and position embeddings. These combinations of preprocessing steps make BERT so versatile. \\

        \item[14.]Flask
        \item[]Flask is a web framework based on Python. You can create simple websites and API servers using Flask. Flask has a higher degree of freedom than Django and is efficient when creating simple services. Python is used a lot to develop images, image processing, and AI-related programs, and Flask is also a much-loved framework at this time. Our team built a backend server using Flask necessary for users to take and analyze photos in the challenge function.\\

        
    \end{enumerate}
\end{enumerate}

\subsection{Software in use}
\begin{enumerate}

    \item[1.]IntelliJ \cite{intellij2011most}
    \begin{figure}[h]
    \centering
    \includegraphics[width=.65\columnwidth]{fig/IntelliJ.png}
    \label{fig:IntelliJ}
    \caption{IntelliJ} 
    \end{figure} 
    \item[]IntelliJ IDEA, JetBrains’ flagship Java IDE, provides high-class support and productivity boosts for enterprise, mobile and web development in Java, Scala and Groovy, with all the latest technologies and frameworks supported out of the box. Every aspect of IntelliJ IDEA is specifically designed to maximize developer productivity. Together, powerful static code analysis and ergonomic design make development a productive and enjoyable experience. Intellij has powerful recommendations, multiple refactoring and debugging capabilities, and supports quick updates tailored to Java and spring boot versions, so we decided to use Intellij to develop backend servers.\\

    \item[2.]Postman \cite{hyams2022you}
    \begin{figure}[h]
    \centering
    \includegraphics[width=.6\columnwidth]{fig/Postman.jpeg}
    \label{fig:Postman}
    \caption{Postman} 
    \end{figure}
    \item[]Postman is a downloadable client and web application that was created as a tool to help with the API testing process. It is now a robust platform for API development, with features to support both the building and use of APIs. Postman has tools for documentation, collaboration with teammates or the larger community, and makes it easy to iterate projects and share them. Postman is a helpful interface that lets you view, send, interact with and use API requests. You can easily see if your request worked and what response was returned. Our team will use postman to express Rest API and resolve any inconveniences that may arise when collaborating between the backend and the frontend.\\ 

    \item[3.]Visual Studio Code
    \begin{figure}[h]
    \centering
    \textbf{}    
    \includegraphics[width=.6\columnwidth]{fig/vscode.png}
    \label{fig:Visual Studio Code}
    \caption{Visual Studio Code} 
    \end{figure}
    \item[]Visual Studio Code (VSCode) is a lightweight, open-source code editor developed by Microsoft, designed for various platforms and primarily used for software development. It supports a wide array of programming languages, providing syntax highlighting, auto-completion, debugging, and language-specific features. Its key strengths include extensibility, allowing users to customize their development environment with numerous extensions. VSCode offers integrated development environment (IDE) features such as code editing, debugging, version control, terminal access, and embedded Git control. It boasts a lightweight installation size, fast startup times, and is supported across Windows, macOS, and Linux operating systems. With a large user community and online support, it fosters collaboration and assistance among users. It is commonly used in web development, application development, data science, and various tech stacks.\\

    \item[4.]Expo
    \begin{figure}[h]
    \centering
    \includegraphics[width=.5\columnwidth]{fig/Expo.png}
    \label{fig:Expo}
    \caption{Expo} 
    \end{figure}
    \item[]Expo is a tool chain built around React Native that streamlines the creation and distribution of cross-platform software. In addition to a managed build environment, and tools for testing and debugging, Expo offers a variety of tools and services that may be used to develop, build, and publish React Native applications. The ability to build features and functionality into their applications makes it simpler for developers to create apps that can work on both Android and iOS. \\

    \item[5.]Figma \cite{staiano2022designing}
    \begin{figure}[h]
    \centering
    \includegraphics[width=.5\columnwidth]{fig/Figma.png}
    \label{fig:Figma}
    \caption{Figma} 
    \end{figure}
    \item[]Figma has succeeded in bringing together a whole suite of design tools to provide an all-in-one solution. Figma covers just about everything you need to create a complex interface, from brainstorming and wireframing to prototyping and sharing assets. In addition to this, Figma goes beyond the design side of building a product and generates CSS, IOS, and Android code for developers to use.\\

    \item[6.]Node.js \cite{tilkov2010node}
    \begin{figure}[h]
    \centering
    \includegraphics[width=.49\columnwidth]{fig/Node.js.png}
    \label{fig:Node.js}
    \caption{Node.js} 
    \end{figure}
    \item[]One of the more interesting developments recently gaining popularity in the server-side JavaScript space is Node.js. It's a framework for developing high-performance, concurrent programs that don't rely on the mainstream multithreading approach but use asynchronous I/O with an event-driven programming model.\\
    
    \item[7.]GitHub
    \begin{figure}[h]
    \centering
    \includegraphics[width=.5\columnwidth]{fig/GitHub-logo.png}
    \label{fig:GitHub}
    \caption{GitHub} 
    \end{figure}
    \item[]GitHub acts as a platform supporting projects utilizing Git, functioning as a remote command center for Git operations. It provides a hub for version control and developer collaboration, operating as a cloud-based version control system. Git and GitHub are commonly used interchangeably for modern software development collaborations, yet GitHub's functionalities extend beyond this scope. To begin, GitHub is the preferred choice for open-source software, granting access to diverse tool source codes used within our team. Moreover, GitHub serves as a repository for identifying issues or bugs in open libraries. Additionally, it boasts various collaborative features: Pull requests enable thorough reviews of work in different Git branches before merging, and GitHub actions streamline the implementation of continuous integration and continuous deployment (CI/CD). Within our team, we employ GitHub actions to monitor team progress and aid in collectively addressing and resolving errors.\\ 

    \item[8.]Notion 
    \begin{figure}[h]
    \centering
    \includegraphics[width=.5\columnwidth]{fig/notion_logo.png}
    \label{fig:Notion}
    \caption{Notion} 
    \end{figure}
    \item[]Notion is a Software as a Service (SaaS) application accessed via the web, operating as a wiki platform. A key benefit is its capability to generate articles in MD file format and provide live updates. With recent enhancements, it has evolved into an invaluable resource for overseeing project details and effectively handling meeting minutes. \\ 
    
    \item[9.]Overleaf \\
    \begin{figure}[h]
    \centering
    \includegraphics[width=.5\columnwidth]{fig/overLeaf.png}
    \label{fig:OverLeaf}
    \caption{OverLeaf} 
    \end{figure}
    \item[]Overleaf serves as an online tool supporting cooperative composition and editing of LaTeX documents. It boasts an intuitive interface specifically designed for the creation of scientific and technical materials like research papers, reports, and theses. Through Overleaf, several team members can work together on a document concurrently, ensuring smooth collaboration and effective monitoring of modifications. Moreover, it integrates pre-installed functionalities for handling references, equations, tables, and graphics, making it a favored option among scholars and researchers. The content of this document was produced using Overleaf's IEEE template. \\ 

    \item[10.]Google Drive \cite{gallaway2013google}
    \begin{figure}[h]
    \centering
    \includegraphics[width=.65\columnwidth]{fig/Google drive.png}
    \label{fig:Google Drive}
    \caption{Google Drive} 
    \end{figure}
    \item[]Google Drive provides two distinct functions. Similar to its forerunner Google Docs, Drive offers online office applications alongside cloud storage, featuring sharing and collaboration capabilities. The recent addition to the platform is a file storage system that synchronizes with a local folder installed by the user, allowing storage and access of various file types through a Google account. The authors analyze the primary functions of Google Drive and evaluate it in comparison to competitors. Furthermore, they explore Google Drive's usefulness for library staff and its role as a collaborative tool supporting students' academic endeavors and education. Our team utilized Google Drive for weekly meetings to monitor the progress of our team members.\\ 

    \item[11.]ChatGPT
    \begin{figure}[h]
    \centering
    \includegraphics[width=.6\columnwidth]{fig/chatGPT.png}
    \label{fig:ChatGPT}
    \caption{ChatGPT} 
    \end{figure}
    \item[]ChatGPT is an AI-driven application that facilitates immediate interactions with an AI. While GPT-3.5 was trained on data until 2021, GPT-4 has been trained on more current information. ChatGPT has transformed generative AI, providing improved functionalities for activities like creating reports, summarizing articles, addressing problems, and even assisting with coding tasks.
    \clearpage
   
    \item[12.]Zoom \cite{kohnke2022facilitating}
    \begin{figure}[h]
    \centering
    \includegraphics[width=.4\columnwidth]{fig/Zoom.png}
    \label{fig:Zoom}
    \caption{Zoom} 
    \end{figure}
    \item[]Zoom, a versatile software platform, serves as a hub for video conferencing, virtual gatherings, and interactive webinars. Through its interface, users can engage in live video and audio communication, fostering seamless online interactions and discussions. It finds widespread application across various domains, encompassing corporate discussions, distance learning, and personal interactions. The platform accommodates extensive virtual gatherings, allowing numerous participants to engage in video calls, utilize chat functions, share screens, and more, ensuring prompt real-time interactions and collaborative efforts. Users can access Zoom through the dedicated app installed on their devices or by using web browsers, providing accessibility across various platforms. Throughout the COVID-19 pandemic, Zoom garnered immense traction due to the surge in remote learning and telecommuting. Nevertheless, its utility extends beyond these realms, serving diverse needs such as professional meetings and everyday communication. Our team opted for Zoom to conduct weekly gatherings.\\ 

    \item[13.]Amazon EC2
    \begin{figure}[h]
    \centering
    \includegraphics[width=.4\columnwidth]{fig/img.png}
    \label{fig:Amazon EC2}
    \caption{Amazom EC2} 
    \end{figure}
    \item[]Amazon Elastic Compute Cloud (Amazon EC2) offers the most comprehensive and in-depth computing platform to best meet the needs of your workload, with over 700 instances and options for the latest processors, storage, networking, operating systems, and purchasing models. AWS is the first major cloud provider to support Intel, AMD, and Arm processors, the only cloud to support an on-demand EC2 Mac instance, and the only cloud to support 400Gbps Ethernet networking. AWS offers the best price/performance for machine learning training, as well as the cheapest rates per inference instance in the cloud. AWS runs more SAP, higher performance computing (HPC), machine learning, and Windows workloads than any other cloud. Our team will distribute the server that we compiled using the spring boot to AWS EC2 to run the server.\\ 

    \item[14.]NUGU PlayBuilder
    \begin{figure}[h]
    \centering
    \includegraphics[width=.4\columnwidth]{fig/nugu_developers_og_img.png}
    \label{fig:NUGU PlayBuilder}
    \caption{NUGU PlayBuilder} 
    \end{figure}
    \item[]NUGU play is a unit of service through the NUGU platform in response to your request, and you can create Play in the Play Builder. Help companies or individuals with good content to provide their services to NUGU users through Play. Create one complete play by combining the User Utterance Model, which understands the user's speech, and the actions that perform functions based on it. Our team uses NUGU speakers for AI communication functions while users use LG Pra.L devices. Using the NUGU playbuilder, three intents are set: skin problem solving, cosmetics recommendation, and daily conversation, and the answer is output in conjunction with openai on the backend proxy server that connects to the NUGU speaker.

\end{enumerate}

\subsection{Cost Estimation}
While creating HI-SKIN, we efficiently developed it with a cost of \$5.5, which was spent on purchasing the OpenAI API key. There were no additional expenses involved.\\ 

\subsection{Task Distribution}

\begin{table}[h]
\centering
\caption{Team Members and Their Tasks}
\renewcommand{\arraystretch}{1.5}
\begin{tabular}{| p{3cm}|p{3cm}|}

\hline
Tasks & Name \\

\hline
Frontend Developer & HAE RYUNG CHA, SEOK YOUNG KIM\\

\hline
Backend Developer & SEOK YOUNG KIM, HAE RYUNG CHA\\

\hline
UI-UX Designer & HAE RYUNG CHA\\

\hline
AI Developer & YU JIN PARK\\

\hline
AI Machine Learning & HAE RYUNG CHA\\

\hline
Product Designer & CHAN MIN KIM\\ 

\hline
\end{tabular}
\end{table}
\clearpage
\section{Specifications}

\subsection{Start Page}
    \begin{figure}[h]
    \centering
    \includegraphics[width=.9\columnwidth]{fig/시작페이지1110.JPG}
    \label{fig:Start Page}
    \caption{Start Page} 
    \end{figure}
When user download the HI-SKIN app and open it for the first time, user will encounter three sequential screens. In the top of each screen, a clean image is displayed, accompanied by a brief description of the app below it. There are two buttons below the photo, along with a bar image. You can navigate forward and backward in the display using the button with an arrow. The current page number is indicated by the image below. Upon reaching the final page, pressing the 'Get Started' button with a black border at the bottom instantly transitions the user to the login screen.\\  

\subsection{Log In}
    \begin{figure}[h]
    \centering
    \includegraphics[width=.43\columnwidth]{fig/HISKIN_231130-04.jpg}
    \label{fig:Log In Page}
    \caption{Log In Page} 
    \end{figure}
This particular page serves as the platform for all actions associated with user login. Within this interface, users are able to log in using their unique ID and password. Positioned at the screen's top section, there exists a designated area where users can input their ID and password. When the user clicks on that area to enter their ID and password, the keyboard will slide up, covering the area below the login button. It is essential that the ID is entered in the correct format, while the password can contain up to 16 characters. When you enter the username and press 'Next' button, it automatically moves to the password input field. \\  Upon accurately entering both the email and password, users can proceed by clicking the login button situated below the password field. Activating this button triggers the transmission of the information in the email and password fields to the backend. The backend then verifies the user's details in the database to ascertain if there's a match and responds accordingly, signaling the success or failure of the login attempt.\\ When user press 'Complete' button or the login button below, it navigates to the home screen. Additionally, for users without an account, there's an option to create one for login purposes by clicking the 'Sign-Up' button located at the bottom of the screen. \\ 

\subsection{Sign Up} \\ \\ \\ 
\begin{figure}[h]
    \centering
    \includegraphics[width=.43\columnwidth]{fig/HISKIN_231130-05.jpg}
    \label{fig:Sign Up Page}
    \caption{Sign Up Page} 
    \end{figure}
Users who do not have an account can create one by selecting the 'Sign Up' button. If user aleready have an account, user can navigate to the log in page by clicking the red 'log in' button located below the 'Sign Up' button. 
\begin{itemize}
    \item NAME (Up to 20 Characters)
    \item Gender (Up to 6 Characters)
    \item NickName (Up to 16 Characters)
    \item ID (Up to 20 Characters)
    \item Password (Up to 16 Characters)
\end{itemize}
When creating a new account, users can freely click on the 'Log in' button to navigate to the login page. \\
Successful Sign Up: Upon receiving new information that does not duplicate an existing ID as an input value, a notification window will display with the message 'You have successfully signed up,' confirming the successful registration. Once the user successfully signs up as a member, they can return to the login screen using the provided button and log in using the information entered during the sign up process.\\

\subsection{Home Page} 
\begin{figure}[hbt!]
    \centering
    \includegraphics[width=.9\columnwidth]{fig/홈1130.JPG}
    \label{fig:Home Page}
    \caption{Home Page} 
    \end{figure}
When the device is not registered, there is nothing above the 'Device Registration' button, but when the device is registered, an image of the device appears. User can register a device by clicking the 'Device Registration' button. Pressing the 'What's My Skin MBTI?' button takes you to the skin type test screen. Additionally, clicking the white button navigates to the challenge screen. If you click on the images of recommended cosmetics, it will take you to the cosmetics recommendation page.\\ \\ \\ \\ \\ \\ \\ \\

\subsection{Device Registration}\\
    \begin{figure}[htb]
    \centering
    \includegraphics[width=.9\columnwidth]{fig/디바이스등록1130.JPG}
    \label{fig:Device Registration Page}
    \caption{Device Registration Page} 
    \end{figure}
The user can register their owned beauty device on the device registration page. An image appears in the middle of the initial screen, accompanied by a message below it: "Please keep your smartphone close to the device."\\
 The screen displays 'loading' during registration. Upon completion of the registration, the message 'registration complete' appears at the middle of the screen, and after a moment, it automatically transitions to the 'My Device' screen.\\
On the 'My Device' screen, details about the device registered by the user are shown, and users can return to the home screen by tapping the 'Complete' button located at the screen's bottom.\\
    
\subsection{Skin Type Test}
Skin type Test is a function that test skin types by selecting answers to skin-related questions and analyzing them. The questions and answers consist of questions selected according to Asian skin characteristics based on the skin type classification method of American dermatologist Leslie Bowman. Questions are largely classified into four categories and each question has two sub-questions. Score according to the user's answer to obtain the sum of the scores of the answers to the two sub-question and give one of the two skin types according to the score. The first question is about dryness and oiliness, and 'D' is given for dryness and 'O' for oiliness. The second question is about sensitivity and resistance, which gives 'S' for sensitivity and 'R' for resistance. The third question is about pigmentation, which gives 'P' for pigmentation and 'N' for non-pigmentation. The fourth question is about wrinkles and tightness, which gives wrinkled skin 'W' and tight skin 'T'. The user would be diagnosed with the final skin type by combining the four diagnosed skin types. It test one of a total of 16 skin types, and information on this is stored in user information, and depending on this skin type, customized skin care solutions such as cosmetics recommendations, challenges, and skin care routine recommendations can be provided. \\ \\
\begin{itemize}
    \item[a.]Questions and Answers
    \begin{figure}[h]
    \centering
    \includegraphics[width=.42\columnwidth]{fig/HISKIN_231130-10.jpg}
    \label{fig:Questions and Answers}
    \caption{Questions and Answers Page} 
    \end{figure}
    \item[]When the user accesses the skin type test screen, user can view questions for the skin type test. Beneath the question, multiple choices are offered for the user to select the most suitable option for their skin type. Clicking on each answer turns the button a light red color. To proceed to the next question, the user can press the 'Next' button at the bottom of the screen. Users receive points corresponding to their selections: 1 point for choice 1, 2 points for choice 2, 3 points for choice 3, 4 points for choice 4, and 2.5 points for choice 5. By aggregating the scores from answers to two sub-questions, the application test one of the two skin types available based on the user's responses. \\ 

    \item[b.]Skin Type Results
    \begin{figure}[h]
    \centering
    \includegraphics[width=.41\columnwidth]{fig/HISKIN_231130-12.jpg}
    \label{fig:Skin Type Results Page}
    \caption{Skin Type Results Page} 
    \end{figure}
    \item[]The user will be diagnosed with their skin type by selecting the last 4-2 question and pressing the ’Next’ button. On the skin type test results page, at the top of the screen, there is a message stating, "'User's Nickname's Skin Type is..." and just below it, there is an image of a clean, clear skin, reminiscent of an egg. Directly beneath this image, the user's skin type is prominently displayed in bold, large font. The overall features of the skin type are located just below the skin type. Below that, it guides you on the specific characteristics of the skin type and the characteristics of cosmetics that fit you well. Press the ’Done’ button located at the bottom of the screen to move the user back to the main page. Skin types are automatically stored in your information, allowing you to receive customized skin care solutions. \\ 
\end{itemize}

\subsection{Cosmetic Recommendations}
Cosmetics Recommendations is a function that allows application to recommend cosmetics that fit the user's skin type. It uses Selenium, a crawling library written in Java on a backend server, to provide cosmetics information obtained from the 'OLIVE YOUNG' site. The best ingredients for each of the 16 skin types are stored in the database, and cosmetics containing good ingredients for each type are searched on the "Olive Young" site to recommend the most popular products. Users will be able to get recommendations for cosmetics that are most optimized for their skin, thereby forming a skincare routine. 
\begin{itemize}
    \begin{figure}[h]
    \centering
    \includegraphics[width=.41\columnwidth]{fig/HISKIN_231130-13.jpg}
    \label{fig:Cosmetic Recommendations Page}
    \caption{Cosmetic Recommendations Page} 
    \end{figure}
    \item[]The picture above is the screen users see when they enter the cosmetics recommendation section. In a chat format, provide information about the user's skin type, explain the characteristics of the type, and recommend products accordingly. \\
\end{itemize}

\subsection{AI Communication}
While receiving care using the beauty device, the user can communicate on three topics. The first topic is skin concerns. When a user talks about skin concerns, the NUGU speaker provides a solution to them. The second topic is cosmetic recommendation. When a user requests a cosmetic recommendation, information on customized cosmetic ingredients and features suitable for the user's skin type is provided. The third topic is daily conversation. We seek to provide users with more pleasure through daily conversations that are related to users’ interests.\\


\subsection{Skin Care Challenge} 
    \begin{figure}[h]
    \centering
    \includegraphics[width=.9\columnwidth]{fig/챌린지1130.JPG}
    \label{fig:Skin Care Challenge Page}
    \caption{Skin Care Challenge Page} 
    \end{figure}
    Challenge scores are given to users by performing the challenge on four skincare routines that are important in daily life selected by Korean Dermatological Association.
    In addition, through skin image analysis using machine learning, users can check their skin conditions. When a user takes a picture of his or her face, AI classifies it into three categories: acne, redness, and dark circles. The application provides customized feedback to the user by combining the challenge score and the face recognition result.
 \\ \\ \\ \\ \\ \\ \\ \\ \\ \\ \\ \\

\subsection{Skin Care}\\ \\ \\
\begin{figure}[h]
    \centering
    \includegraphics[width=.9\columnwidth]{fig/캐어1130.JPG}
    \label{fig:Skin Care Page}
    \caption{Skin Care Page} 
    \end{figure}
This page provides a voice guide that makes a user feel like the user is getting care from a dermatologist when the user takes care of your skin. By checking the recent records, user can view a list of recent care activities. Below that, there is a list of connected devices, and clicking on each device divides the care activities by device. When selecting each care activity, there is a circle animation in the center that moves according to the voice of the AI, along with displayed connected speaker and remaining time at the bottom.\\

\subsection{Community} 
    \begin{figure}[h]
    \centering
    \includegraphics[width=.42\columnwidth]{fig/HISKIN_231130-21.jpg}
    \label{fig:Community Page}
    \caption{Community Page} 
    \end{figure}
HI-SKIN has a community menu included in the bottom tab. The user can check the entire article by entering the community menu. The article guides useful information such as how to use the device and secret to skin care.


\subsection{My Page}
    \begin{figure}[h]
    \centering
    \includegraphics[width=.35\columnwidth]{fig/HISKIN_231130-22.jpg}
    \label{fig:My Page 1}
    \caption{My Page 1} 
    \end{figure}
When the user enters the My Page screen, they can access comprehensive information about their skin. \\

\begin{figure}[h]
    \centering
    \includegraphics[width=.8\columnwidth]{fig/내정보113-.JPG}
    \label{fig:My Page 2}
    \caption{My Page 2} 
    \end{figure}
    Clicking on the 'Skin Type Settings' button takes you to the skin type test screen, allowing you to retake the skin type test. Below that, clicking on the 'Skin Report' button enables you to review the user's skin challenge records for 1 week, 3 months, and 6 months.\\
    On the skin report page, users can view their challenge score graph. At the top of the screen, pressing buttons for one week, three months, and six months allows users to check the respective challenge reports. Below these buttons, the challenge scores are graphically represented. Users can easily observe fluctuations in challenge scores through this graph. Beneath the graph, the challenge score records are presented. For the one-week report, days of the week are listed, while for the three-month and six-month reports, dates are provided, each accompanied by the corresponding challenge score. By clicking the top-left back arrow button, users can return to the My Page.\\ \\

\section{Architecture Design Implementation}
\subsection{Overall Architecture}
The overall structure of our service consists of an application and AI Nugu speaker. \\ 
\begin{figure}[h]
    \centering
    \includegraphics[width=.8\columnwidth]{fig/1234.png}
    \label{fig:Application Framwork}
    \caption{Application Framwork} 
    \end{figure} \\ 
\indent The first module that makes up the application is the frontend. Our team proceeded with development using React Native and expo. Users can perform skin type tests through the application and receive cosmetic recommendations. You can check the community page where application users can share their own beauty tips, and on the skincare challenge screen, you can check your own challenge score and a graph that analyzes your skin challenge for three months and six months.\\

The second module that makes up the application is the backend. A spring boot was used to build an application server, and hibernate was used as a database linked to it. The backend serves to perform the tasks requested by the user. Our application performs the functions of signing up and logging in, calculating skin type test results, recommending cosmetics, and calculating challenge scores. A user table, a skin type table, a challenge score table, and a challenge response table exist in the database. When the information is received from the user, the server stores the information in the database.\\

Another structure of our service is the Nugu speaker. When users take care of their skin using an LG Pra.L device, they can talk to Nugu speakers. The Nugu speaker was developed using the Nugu playbuilder provided by SKT, the AI backend linked to Nugu used Google Cloud Platform, and the language used node.js. There are a total of three intents in the conversation: solving skin problems, recommending cosmetics, and daily conversations. When the user talks to the speaker, it analyzes which of the three intentions are and responds to the user by outputting the data value received from the backend from the speaker among the various actions accordingly.

\begin{figure}[h]
    \centering
    \includegraphics[width=.8\columnwidth]{fig/ERD.png}
    \label{fig:Database Entity Relationship Diagram}
    \caption{Database Entity Relationship Diagram} 
    \end{figure}

\subsection{Directory Organization}
HI-SKIN consists of four GitHub repositories: HISKIN\_frontend, HISKIN\_backend, HISKIN\_AI, and HISKIN\_documentation. The HISKIN\_frontend repository contains files related to overall design and functions for interacting with the application's users. The HISKIN\_backend repository includes files that work with the repository and database. The HISKIN\_AI repository contains files related to the AI component of HI-SKIN. Lastly, the HISKIN\_documentation repository includes Latex code and a PDF file documenting the project.  \\
\begin{figure}[h]
    \centering
    \includegraphics[width=\columnwidth]{fig/13241234.JPG}
    \label{fig:Repository}
    \caption{Repository} 
    \end{figure}

\subsection{Module 1: Front-End}
\begin{itemize}
    \item [1.] Purpose: We have developed a React Native-based skincare management app for users of LG Pra.L products. React Native allows for cross-platform development and boasts a rich set of libraries. Additionally, we utilized Expo, enabling us to display and modify code in real-time on actual devices. Through this app, we establish a connection between users and the backend, fetching data from the backend to present it to users in real-time. \\ 
    \item[2.]Functionality: React Native facilitates communication between users and various components, as well as the backend, through the app's UI/UX. The frontend is developed using React, while the backend handles functionalities such as login, sign-up, skin type testing (with options for selecting skin types), skin type results, and challenges (with scores for each category). Throughout this process, the server connects this data with user information and stores it in the database.
    \\
    \item[3.] Location of Source Code: HYU-SE-HISKIN/HISKIN\_frontend 
    
    
    \begin{table} [htb]
    \begin{tabular}{p{2.5cm}|p{3.5cm}|p{2cm}}
    \hline
    \textit{\textbf{Directory}} & \textit{\textbf{File Name}} & \textit{\textbf{Modules Used}}\\ 
    \hline
    \begin{tabular}[c]{@{}l@{}}HISKIN\_frontend\end{tabular} & \begin{tabular}[c]{@{}l@{}}App.js\\app.json\\babel.config.js\\metro.config.js\\package-lock.json\\yarn.lock\end{tabular} 
    & \begin{tabular}[c]{@{}l@{}}\end{tabular} \\ 
    
    \hline
    \begin{tabular}[c]{@{}l@{}}HISKIN\_frontend/\\assets/images\end{tabular} & \begin{tabular}[c]{@{}l@{}}welcome1.png\\welcome2.png\\welcome3.png\\welcomeLayer1\\welcomeLayer2\\welomeLayer3\\bar1.png\\bar2.png\\bar3.png\\AppName\_small.svg\\AppName\_large.svg\\mydevice.svg\\iconhome.svg\\iconchallenge.svg\\iconcare.svg\\iconcommunity.svg\\iconmyinfo.svg\\iconhome\_filled.svg\\iconcare\_filled.svg\\iconcommunity\_filled.svg\\iconid.svg\\iconpw.svg\\IconAccount.svg\\IconReport.svg\\IconSetting.svg\\IconHuman.svg\\skintyperesult.svg\\editbutton.svg\\plusbutton.svgnextbutton.svg\\morebutton.svg\\heart.svg\\communityimage.svg\\cosmetic1.svg\\cosmetic2.svg\\cosmetic3.svg\\CareMain.svg\\CareRoutine1.svg\\CareRoutine2.svg\\CareRoutine3.svg\\checkbox\_unchecked.svg\\checkbox\_checked.svg\end{tabular} 
    & \begin{tabular}[c]{@{}l@{}} \end{tabular} \\ 

    
    \hline
    
    \end{tabular}
\end{table}
\clearpage

\begin{table} [h]
    \begin{tabular}{p{2.5cm}|p{3.5cm}|p{2cm}}
    \hline
    
    \begin{tabular}[c]{@{}l@{}}HISKIN\_frontend\\assets/fonts\end{tabular} & \begin{tabular}[c]{@{}l@{}}LGSmartBoldItalic.ttf\\LGSmartBold.ttf\\LGSmartLight.ttf\\LGSmartRegularItalic.ttf\\LGSmartRegular.ttf\\LGSmartSemiBold.ttf\\LGEIHeadlineTTF-Bold.ttf\\LGEIHeadlineTTF-Light.ttf\\LGEIHeadlineTTF-Regular.ttf\\LGEIHeadlineTTF-Semibold.ttf\\LGEIHeadlineTTF-Thin.ttf\\LGEIHeadlineVF.ttf\\LGEITextTTF-Bold.ttf\\LGEITextTTF-Light.ttf\\LGEITextTTF-Regular.ttf\\LGEITextTTF-SemiBold.ttf\\Calistoga-Regular.ttf\end{tabular} 
    & \begin{tabular}[c]{@{}l@{}}\end{tabular} \\ 

    \hline
    \begin{tabular}[c]{@{}l@{}}HYU-SE-HISKIN/\\HISKIN\_frontend\\src\end{tabular} & \begin{tabular}[c]{@{}l@{}}App.js\\theme.js\end{tabular} 
    & \begin{tabular}[c]{@{}l@{}}\end{tabular} \\

    \hline
    \begin{tabular}[c]{@{}l@{}}HISKIN\_frontend\\src/components\end{tabular} & \begin{tabular}[c]{@{}l@{}}Button.js\\CheckBox.js\\CheckBoxContainer.js\\ChoiceContainer.js\\CommunityBox.js\\EmptyBox.js\\HorizontalContainer.js\\HyperLinkText.js\\Image.js\\ImageLinker.js\\Input.js\\Loading.js\\OptionButton.js\\PlusButton.js\\Graph.js\\IvoryContainer\\WhiteContainer\end{tabular} 
    & \begin{tabular}[c]{@{}l@{}}\end{tabular} \\ 

    \hline
    \begin{tabular}[c]{@{}l@{}}HISKIN\_frontend\\src/contexts\end{tabular} & \begin{tabular}[c]{@{}l@{}}User.js\end{tabular} 
    & \begin{tabular}[c]{@{}l@{}}\end{tabular} \\ 

    \hline
    \begin{tabular}[c]{@{}l@{}}HISKIN\_frontend\\src/navigations\end{tabular} & \begin{tabular}[c]{@{}l@{}}AuthStack.js\\MainStack.js\\MainTab.js\end{tabular}
    & \begin{tabular}[c]{@{}l@{}}\end{tabular} \\ 

    \hline
    \begin{tabular}[c]{@{}l@{}}HISKIN\_frontend\\src/screens\end{tabular} & \begin{tabular}[c]{@{}l@{}}Care.js\\Challenge.js\\Community.js\\Cosmetics.js\\DeviceRegistration.js\\Home.js\\Info.js\\Login.js\\Signup.js\\SkinReport.js\\SkinTypeResult.js\\SkinTypeTest.js\\Welcome.js\\CareDetails.js\\FacialAnalysis.js\\InteractionCare.js\\TakePhoto.js\end{tabular}
    & \begin{tabular}[c]{@{}l@{}}\end{tabular} \\ 

    \hline
    \begin{tabular}[c]{@{}l@{}}HISKIN\_frontend\\src/utils\end{tabular} & \begin{tabular}[c]{@{}l@{}}skinTypeQuestions.js\\skinTypeResults.js\\welcomeMessages.js\\facialAnalysisResults.js\end{tabular}
    & \begin{tabular}[c]{@{}l@{}}\end{tabular} \\ 
    
    \hline
    \end{tabular}
\end{table}

\item[4.] Class Components:
\begin{itemize}
    \item[$\bullet$] AuthStack: Within the stack, there are Welcome, Login, and Signup screens.
    \item[$\bullet$] App: During the app loading process, all fonts are loaded from assets/fonts. Once this operation is complete, the components necessary to represent the app are rendered on the screen.
    \item[$\bullet$] Care: At the top, users can view a history of the devices they have used. Below that, a list of registered devices is displayed. Selecting a device reveals the care routine for that device, and clicking on the routine allows users to initiate the care process. The central image's animation changes according to the selected care routine, aligning with the chosen care.
    \item[$\bullet$] CareDetails.js: When the user selects a device on the care screen, they are directed to CareDetail.js. The chosen device name is displayed, and there are buttons to select routines specific to that device. Upon selecting each routine, the user is navigated to InteractiveCare.js with the routine name being passed along.
     \item[$\bullet$] Challenge: Users can measure their daily skin score, divided into four areas, each with checkboxes containing data ranging from 5 to 25 points. Clicking these checkboxes fills the circular line graph at the top in real-time, allowing users to visually check their scores. These scores are transmitted to the backend and recorded daily.
    \item[$\bullet$] Cosmetics: When a user completes the skin type test, the results are stored in their user information and utilized for the cosmetic recommendation feature. If there are no skin type results in the backend, it notifies the user of an error. The skin type results are sent to the backend to initiate web scraping using the 'ChromeDriver.' It searches the Olive Young website for key ingredients corresponding to each skin type and provides the front end with the image and name of the first product that appears. The front end presents this data to the user in a chat format to enhance their understanding of the information.
    \item[$\bullet$] Community: A screen where users can share simple daily concerns, skin issues, or express themselves through text or attached photos.
    \item[$\bullet$] DeviceRegistration: After the loading page, a message appears saying "The device has been registered," accompanied by an image of the registered device on the screen.
    \item[$\bullet$] FacialAnalysis.js: In TakePhoto.js, the received photo is sent to the Flask server. A loading page is displayed until a response is received. Once a response is received, the facial recognition results are displayed according to the response, which may include information about acne, redness, and dark circles.
     \item[$\bullet$] Info: There are buttons that allow users to either take the skin type test or access their skin score report. Pressing the "Skin Type Test" button navigates to the skin type test screen, while clicking the "Report" button leads to the Skin Report screen.
    \item[$\bullet$] InteractionCare.js: The received routine name is displayed at the top. In the middle of the screen, there is a color-changing circular animation, and below that, the remaining time for the routine is shown to help the user focus on their care.
     \item[$\bullet$] Login: Users can input their ID and password. When they enter the data, the moment they press the 'Log In' button, the values are sent to the backend. If the backend database contains the information, the login is successful, and the user is directed to the home screen. If the user information is not found, they can navigate to the sign-up screen by clicking on the 'Sign Up' text at the bottom.
   
    
    \item[$\bullet$] MainStack: Within the stack, there are MainTab as well as DeviceRegistration, Care, Challenge, SkinTypeTest, SkinTypeResult, Cosmetics, and SkinReport screens.
    \item[$\bullet$] MainTab: Within the stack, there are Home, Challenge, Care, Community, and Info screens.
    
   
    \item[$\bullet$] Signup: Users can enter their name, gender, nickname, ID, and password. If they press the 'Sign Up' button, the data is sent to the backend and stored in the database. If there are no issues in this process, the user is immediately directed to the home screen with a successful sign-up message. The user can then log in using this ID and password. By clicking on the login text at the bottom of the sign-up page, they can return to the login screen.
    \item[$\bullet$] Home: After a successful login, the user is greeted with the main screen containing an overview of the app's functionalities. Various buttons provide access to device registration, skin type analysis, challenges, and cosmetic recommendations. At the bottom, users can also view community posts. If no devices are registered, users can press the "Device Registration" button to register a device. After registration, the device's appearance is displayed below the app logo. Clicking on the device picture allows users to navigate to the care screen. By clicking on the "Skin Type Test" button, users are directed to the skin type testing page. The "Challenge" button leads to the challenge screen. Clicking on the text "Cosmetic Recommendations" or cosmetic images takes users to the cosmetic recommendations page.
    \item[$\bullet$] SkinReport: Clicking the "Report" button allows users to view graphs of their skin scores over one week, three months, and six months. Clicking the "Duration" text and the buttons for one week, three months, or six months displays the corresponding graph in the white box below. Skin score data is retrieved from the backend upon request.
    
    \item[$\bullet$] SkinTypeTest: The app displays 8 questions and 4 answer choices from `utils/skinTypeQuestions.js` on the screen. Upon selecting an answer choice, the choice turns red, and clicking the 'Next' button transitions to the next question. On the final 8th question, clicking the 'Complete' button sends the selected choices to the backend. The app then receives the examination results from the backend, and these results are passed to the `SkinTypeResult` screen, which is navigated to upon clicking 'Complete.'
    \item[$\bullet$] SkinTypeResult: The app displays the conveyed skin type results and corresponding explanations on the screen. Users can use this screen to gain a better understanding of their skin and receive recommendations for cosmetic ingredients suitable for their skin.
   \item[$\bullet$] TakePhoto.js: This is the screen where users can take photos using the camera within the app. Users can switch between the rear and front camera types, and there is a bottom center button to capture the photo. Once a photo is taken, it is displayed on the screen, and users have the option to save the image or take another one. If saved, the image is sent while navigating to FacialAnalysis.
    \item[$\bullet$] User:  Using `useContext`, we can check whether the user has logged in or signed up, confirming whether the username and password information has been sent to the backend. If the information has been successfully transmitted, we can then immediately navigate from the AuthStack to the MainStack.
    \item[$\bullet$] Welcome: It delivers the overall description of the HI-SKIN application in card format. Users can refer to the design, function, and structure of the application as they turn over these cards.
    \\ 
\end{itemize}

\item[5.] Where It's Taken From: The user interface on the frontend provides a direct gateway for users to interact with the HISKIN service. Users input essential information to access the platform. Furthermore, utilizing the data repository, the interface seamlessly showcases personalized content through an array of visual materials. \\ 

\item[6.] How / Why you Used it: React offers the advantage of easy installation, allowing for the straightforward use of various modules. Additionally, the abundance of examples and a broad community provided ample reference materials for development. Using Expo allowed us to test our application on both Android and iOS seamlessly. The ability to view and modify the screens in real-time during testing further facilitated our development process. Hence, we chose React and Expo for our development. \\
\end{itemize}


\subsection{Module 2: Back-End}
\begin{itemize}
    \item [1.] Purpose: The backend server receives a request from the frontend, performs work in conjunction with the database, and delivers the value back to the frontend. When a user makes a request through an app, the server performs it, and when a value is given, the value is stored in the database. Our team created a backend server using Springboot, a representative framework of Java used by many domestic companies as a development language, and made Amazon EC2 servers operate in a Ubuntu environment. As the database, Hibernate, which is known to be very compatible with spring boots, is used. Build a backend server by storing information necessary for users to use the application in Hibernate. \\ 
    \item[2.] Functionality: The server performs a function of performing a task requested by the user and storing a value corresponding thereto in a database. When a user signs up for membership, information is stored in a database, and when a skin type test is performed, the user's skin type is stored and cosmetics that match it are recommended. It also performs the function of storing scores and time in the database and sending data of the last seven, three, and six months to the frontend whenever the user performs the challenge.\\
    
    \item[3.] Location of Source Code: HYU-SE-HISKIN/HISKIN\_backend 
    
    \begin{table} [hbt!]
    \begin{tabular}{p{2.5cm}|p{3.5cm}|p{2cm}}
    \hline
    \textit{\textbf{Directory}} & \textit{\textbf{File Name}} & \textit{\textbf{Modules Used}}\\ \hline
    \begin{tabular}[c]{@{}l@{}}HISKIN\_backend\end{tabular} & \begin{tabular}[c]{@{}l@{}}.github.workflows\\scripts\\src/main\\appspec.yml\\build.gradle\end{tabular} 
    & \begin{tabular}[c]{@{}l@{}}\end{tabular} \\ \\
    
    \hline 
    \begin{tabular}[c]{@{}l@{}}HISKIN\_backend/\\.github/workflows\end{tabular} & \begin{tabular}[c]{@{}l@{}}\\deploy.yml\\ \\ \end{tabular} 
    & \begin{tabular}[c]{@{}l@{}}\end{tabular} \\ \\ 
    
    \hline 
    \begin{tabular}[c]{@{}l@{}}HISKIN\_backend/\\scripts\end{tabular} & \begin{tabular}[c]{@{}l@{}}\\start.sh\\ stop.sh \\ \\ \end{tabular} 
    & \begin{tabular}[c]{@{}l@{}}authController.js\\authRouter.js\\authService.js\end{tabular} \\ \\ 
    
    \hline 
    \begin{tabular}[c]{@{}l@{}}HISKIN\_backend/\\src/main\end{tabular} & \begin{tabular}[c]{@{}l@{}}java/hiskin\_hiskin\_backend\\resources\end{tabular} 
    & \begin{tabular}[c]{@{}l@{}}\end{tabular} \\ \\
    
    \hline 
    \begin{tabular}[c]{@{}l@{}}HISKIN\_backend/\\src/main/resources\end{tabular} & \begin{tabular}[c]{@{}l@{}}application.yml\end{tabular} 
    & \begin{tabular}[c]{@{}l@{}}\end{tabular} \\
    
    \hline 
    \begin{tabular}[c]{@{}l@{}}HISKIN\_backend/\\src/main/java/\\hiskin\_hiskin\_backend\end{tabular} & \begin{tabular}[c]{@{}l@{}}config\\controller\\domain\\dto\\repository\\service\\util\\HiskinBackendApplication.java. \end{tabular} 
    & \begin{tabular}[c]{@{}l@{}}SpringBoot\\Application\end{tabular} \\
    
    \hline
    \begin{tabular}[c]{@{}l@{}}HISKIN\_backend\\/src/main/java/\\hiskin\_hiskin\_backend/\\config\end{tabular} & \begin{tabular}[c]{@{}l@{}}CorsConfig.java\end{tabular} 
    & \begin{tabular}[c]{@{}l@{}}Bean\\Configuration\\CorsRegistry\\WebMvcConfigurer \end{tabular} \\
    
    \hline
    \begin{tabular}[c]{@{}l@{}}HISKIN\_backend/\\src/main/java/\\hiskin\_hiskin\_backend/\\controller\end{tabular} & \begin{tabular}[c]{@{}l@{}}ChallengeController.java\\CosmeticsController.java\\MyPageController.java\\SkinTypeController.java\\UserController.java\\SkinTypeController.java\end{tabular} 
    & \begin{tabular}[c]{@{}l@{}}RestController\\RequestMapping\\GetMapping\\PostMapping\\Autowired\\HttpStatus\\MediaType\\ResponseEntity\\LocalDate\\HashMap\\List\\Map\\Collectors \end{tabular} \\

    \hline
    \begin{tabular}[c]{@{}l@{}}HISKIN\_backend/\\src/main/java/\\hiskin\_hiskin\_backend/\\domain\end{tabular} & \begin{tabular}[c]{@{}l@{}}ChallengeScore.java\\User.java\end{tabular} 
    & \begin{tabular}[c]{@{}l@{}}lombol\\persistence\\Getter\\Setter\\NoArgsConstructor\end{tabular} \\

    \hline
    
    \end{tabular}
\end{table}


    \begin{table} [htb!]
    \begin{tabular}{p{2.5cm}|p{3.5cm}|p{1.5cm}}

    \hline
    \begin{tabular}[c]{@{}l@{}}HISKIN\_backend/\\src/main/java/\\hiskin\_hiskin\_backend/\\dto\end{tabular} & \begin{tabular}[c]{@{}l@{}}ChallengeResponseDTO.java\\ChallengeScoreDTO.java\\SkinTypeResponse.java\\UserLoginRequest.java\\UserRegistrationRequest.jav\end{tabular} 
    & \begin{tabular}[c]{@{}l@{}}Getter\\Setter\\LocalDate\end{tabular} \\
    
    \hline
    \begin{tabular}[c]{@{}l@{}}HISKIN\_backend/\\src/main/java/\\hiskin\_hiskin\_backend/\\repository\\ \end{tabular} & \begin{tabular}[c]{@{}l@{}}ChallengeScoreRepository.java\\UserRepository.java\end{tabular} 
    & \begin{tabular}[c]{@{}l@{}}LocalDate\\List\\JpaRepository\\Modifying\\Query\\Param\end{tabular} \\
    
    \hline
    \begin{tabular}[c]{@{}l@{}}HISKIN\_backend/\\src/main/java/\\hiskin\_hiskin\_backend/\\service\end{tabular} & \begin{tabular}[c]{@{}l@{}}ChallengeScoreService.java\\ChallengeService.java\\CosmeticsCrawlerService.java\\SkinTypeTestService.java\\UserService.java\\ \end{tabular} 
    & \begin{tabular}[c]{@{}l@{}}Autowired\\Service\\Selenium\\HashMap\\Map\\PageRequest\\Pageable\\Sort\end{tabular} \\ \\

    \hline
    \begin{tabular}[c]{@{}l@{}}HISKIN\_backend/\\src/main/java/\\hiskin\_hiskin\_backend/\\util\end{tabular} & \begin{tabular}[c]{@{}l@{}}ChallengeScoreGenerator.java\\LoggedInUserHolder.java\\SkinTypeSearchKeywords.java\end{tabular} 
    & \begin{tabular}[c]{@{}l@{}}ArrayList\\List\\Random\\Component\end{tabular}\\ \\ 
    \hline
    \end{tabular} 
\end{table}

    \item[4.]  Class Components: 
    \begin{itemize}
        \item[$\bullet$] build.gradle: This is a file that builds the dependencies and settings needed for the spring boot to work.
        \item[$\bullet$] appspec.yml: this file is a file that Code Deploy will refer to for deployment to AWS EC2. With the AppSpec file, we can set which files in the project are copied to which path in EC2, and we can also automatically launch the server by specifying a script to perform after the deployment process.
        \item[$\bullet$] .github/workflows/deploy.yml: When the file is pushed to the GitHub main branch, the github action flow is activated to push it to the AWS S3 bucket and then perform CodeDeploy
        \item[$\bullet$] scripts/start.sh: As a script to run the application, only copy and run the JAR file because we have already built it in the GitHub Actions workflow.
        \item[$\bullet$] scripts/stop.sh: Script to exit if the application is already up.
        \item[$\bullet$] CorsConfig: This class is a code that resolves CORS errors when working with backend servers and frontend servers. All Get, Post, Put, and Delete methods at the address "http://localhost:8081" are allowed to solve CORS-related problems that arise when the backend and frontend are linked.
        \item[$\bullet$] ChallengeController: This class is a file related to the skin care challenge API. When a user performs a challenge, it includes an API that calculates the total score of the challenge and an API that responds to the frontend of the last seven challenge scores.
        \item[$\bullet$] CosmeticsController: This class is a file related to the cosmetic recommendation API. It includes an API that inquires the skin type stored in the user DB, searches for a search word matching it on the OliveYoung site, and responds to the frontend with the product name and image url ranked first in popularity.
        \item[$\bullet$] SkinTypeController: This class is a file realted to the skin type API. When a request for the skin type is made in the frontend, the class retrieves and returns the information by querying the user datavase
        \item[$\bullet$] MyPageController: This class is a file related to the My Page Skin Challenge cumulative data API. When the user inquires about the skin challenge situation of 3 or 6 months, the response is provided to the frontend.
        \item[$\bullet$] SkinTypeTestController: This class is a file related to the skin type test API. When a user proceeds with eight skin type tests, the score is analyzed and the result of one of the 16 skin types is provided to the frontend.
        \item[$\bullet$] UserController: This class is a file related to user registration and login API. It includes an API that provides a response to the success or failure of the user's registration and an API that provides a response to the success or failure of the login compared to the user DB at the time of login.
        \item[$\bullet$] ChallengeScore: This class is a file related to the "Challenge Score" table. The challenge score db column includes userId, score, and date.
        \item[$\bullet$] User: This class is a file associated with the "users" table. Columns in user db include name, nickname, gender, userId, password, skin type, and challenge score.
        \item[$\bullet$] ChallengeResponseDTO: This class is a file related to storing the responses of the four challenges.
        \item[$\bullet$] ChallengeScoreDTO: This class is a file related to the challenge score and date.
        \item[$\bullet$] SkinTypeResponse: This class is a file related to user's skin type
        \item[$\bullet$] UserLoginResponse: This class is a file related to login process. It includes userID and password
        \item[$\bullet$] UserREgistrationRepository: This class is a file related to registration process. It includes name, nickname, gender, userId and password.
        \item[$\bullet$] ChallengeScoreRepository: This class is a file that makes the database accessible using JPA in relation to the challenge score.
        \item[$\bullet$] UserRepository: This class is a file that makes the database accessible using JPA in relation to the users.
        \item[$\bullet$] ChallengeScoreService: This class is a file related to processing the challenge score. It includes a function that stores the challenge score and date in db whenever the user performs the challenge, and a function that extracts the recent seven challenge scores.
        \item[$\bullet$] ChallengeService: This class is a file related to how to calculate the challenge score. When the user performs the challenge, it includes a function that calculates the total score by assigning a score according to the answer. In addition, when the challenge is performed, the function of storing the challenge score in the user db is also included.
        \item[$\bullet$] CosmeticsCrawlerService: This class is a file related to how to recommend cosmetics. It includes a function of searching for a skin type from the user's db and searching for a search word matching it on the Olive Young site to crawl the top product name and image urld in the popularity order.
        \item[$\bullet$] SkinTypeTestService: This class is a file related to skin type testing. It includes a function that finally diagnoses the final skin type by changing the answer selected by the user in the skin type test into a score and assigning skin types for a total of four skin categories.
        \item[$\bullet$] UserService: This class is a file related to the user. It includes a function of storing the skin type in the user DB when the user proceeds with the skin type test and a function of storing the challenge score in the user DB when performing the challenge.
        \item[$\bullet$] ChallengeScoreGenerator: This Class is a file that generates a challenge score.
        \item[$\bullet$] LoggedInUserHolder: This class is a file that stores userID when a user logs in.
        \item[$\bullet$] SkinTypeSearchKeywords: This class is a file containing search keys to be searched on Olive Young sites for each skin type.
        \item[$\bullet$] src/maim/resources/application.yml: This class is a file related to the setting for the spring boot to work properly. \\
    \end{itemize}
    
    \item[5.] Where It's Taken From: The backend server works with the database while exchanging values from the front end. HI-SKIN's backend server is currently running on AWS EC2 because it's a good platform to run the server for a small amount of money.
\\ 
    \item[6.] How / Why you used it: The backend server performs tasks that need to happen when the user uses the app. It aims to provide satisfactory services to users while storing and importing values in the database. Among many development languages, servers were built using Springboot, a representative framework of Java known to be widely used by Korean companies.
\end{itemize}

\subsection{Module 3: Back-end for Facial Analysis}
\begin{itemize}
    \item [1.] The backend server, built using Flask and written in Python, performs tasks and returns appropriate values in response to requests from the frontend. The backend is well-suited for running machine learning models due to its Python-based code. \\
    
    \item[2.] Functionality: When the user posts a photo taken on the TakePhoto page, it is received, adjusted to fit the model's requirements, and then processed through the model. Subsequently, the resulting values are conveyed in JSON format.

`   \item[3.] Location of Source Code: HYU-SE-HISKIN/HISKIN\_backend\_FacialAnalysis\\

\begin{table} [h]
    \begin{tabular}{p{3cm}|p{3cm}|p{1.7cm}}
    \hline
    \begin{tabular}[c]{@{}l@{}}HISKIN\_backend\_\\FacialAnalysis\end{tabular} & \begin{tabular}[c]{@{}l@{}}app.py\\best\_model.pth\\requirement.txt\end{tabular}  & \begin{tabular}[c]{@{}l@{}}\end{tabular} \\ 
    \hline
    \end{tabular} 
\end{table}

    \item[4.] Class Components
    \begin{itemize}
        \item[$\bullet$]requirements: It contains information about the version of libraries used in the project.
        \item[$\bullet$]best model.pth: This file implements a ViT (Vision Transformer) model using PyTorch, and it represents the saved state when the loss is minimized. Utilizing this file allows for rapid inference of results when provided with input images.
        \item[$\bullet$]app.py: It contains information about the server, defining processes for data input, data processing, and responses for each server route. It also outlines the data preprocessing steps before feeding it into the model.\\
    \end{itemize}

    \item[5.] Where It's Taken From: The Flask server enables users to POST photos, allowing these photos to be fed into a machine learning model.
\\ 
    \item[6.] How / Why you used it: Flask makes it easier to access machine learning models written in Python, as it is based on Python.

\end{itemize}
    
    

\subsection{Module 4: AI Communication}
\begin{itemize}
    \item [1.] Purpose: We wanted to give application users the feeling of being treated at a dermatologist while using beauty devices. Therefore, we tried to implement a communication function that allows conversations on three topics. \\
    
    \item[2.] Functionality: AI communication can be divided into two main parts. They are OpenAI and NLP. We implemented a chat model using OpenAI API for the natural response, and we created a chatbot model using a transformer among natural language processing models to have a conversation about everyday life. \\
    First, in the case of OpenAI, based on the user utterance information transmitted from the backend, the speaker provides an appropriate response using OpenAI's chat completion API. This can be customized by limiting the system to answer questions in the desired field. In addition, even though it is the user's first question, there is an advantage that the conversation details with the user can be stored as a template to respond appropriately to the user's request.\\
    Furthermore, the chatbot model implemented the ability to have daily conversations with AI using the transformer model. To this end, we looked at the core components of the transformer model, such as positional encoding, scaled-dot product attention, multi-head attention, feedforward network, layer normalization, and encoder/decoder modules, and implemented the entire architecture of the transformer integrated with them. In the chatbot model, when a user's input is inputted as a string of Python, ‘preprocess\_sentence’ function preprocesses the string. For a preprocessed string, ‘evaluate’ function delivers the preprocessed user's input to the transformer model and sequentially predicts the words corresponding to the chatbot's answer through the decoder. ‘Predict’ function decodes the integer sequence corresponding to the chatbot's answer received from the evaluate function as a string again and outputs the chatbot's answer to the user.
\\
    
    \item[3.] Location of Source Code: HYU-SE-HISKIN/HISKIN\_AI\\

    
    \begin{table} [h]
    \begin{tabular}{p{3cm}|p{3.3cm}|p{1.7cm}}
    \hline
    \begin{tabular}[c]{@{}l@{}}HISKIN\_AI/\\Openai Model\end{tabular} & \begin{tabular}[c]{@{}l@{}}SE\_FineTunning.ipynb\\SE\_LangChain\_Basic.ipynb\\SE\_LangChain\_Agent.ipynb\\SE\_LangChain_Basic.ipynb\\SE\_OpenAI\_API.ipynb\end{tabular}  & \begin{tabular}[c]{@{}l@{}}\end{tabular} \\ 

    \hline
    \begin{tabular}[c]{@{}l@{}}HISKIN\_AI/\\nlp\end{tabular} & \begin{tabular}[c]{@{}l@{}}data\\models\\Chatbot\_Tansformer.ipynb\end{tabular} 
    & \begin{tabular}[c]{@{}l@{}}\end{tabular} \\ 

    \hline
    \begin{tabular}[c]{@{}l@{}}HISKIN\_AI/\\nlp/models\end{tabular} & \begin{tabular}[c]{@{}l@{}}Basic Embedding Model\\CNN\\RNN\\Attention Mechanism\\ Transformer\end{tabular} 
    & \begin{tabular}[c]{@{}l@{}}\end{tabular} \\ \\

    \hline
    \begin{tabular}[c]{@{}l@{}}HISKIN\_AI/\\nlp/models/\\Basic Embedding Model\end{tabular} & \begin{tabular}[c]{@{}l@{}}NNM.ipynb\\Word2Vec.ipynb\\FastText\end{tabular} 
    & \begin{tabular}[c]{@{}l@{}}\end{tabular} \\ 

    \hline
    \begin{tabular}[c]{@{}l@{}}HISKIN\_AI/\\nlp/models\\CNN\end{tabular} & \begin{tabular}[c]{@{}l@{}}TextCNN.ipynb\end{tabular} 
    & \begin{tabular}[c]{@{}l@{}}\end{tabular} \\ 

    \hline
    \begin{tabular}[c]{@{}l@{}}HISKIN\_AI/\\nlp/models/\\RNN\end{tabular} & \begin{tabular}[c]{@{}l@{}}Bi-LSTM.ipynb\\TextLSTM.ipynb\\TestRNN.ipynb\end{tabular} 
    & \begin{tabular}[c]{@{}l@{}}\end{tabular} \\ 

    \hline
    \begin{tabular}[c]{@{}l@{}}HISKIN\_AI/\\nlp/models/\\Attention Mechanism\end{tabular} & \begin{tabular}[c]{@{}l@{}}BI-LSTM(Attention).ipynb\\Seq2Seq(Attention).ipynb\\Seq2Seq.ipynb\end{tabular} 
    & \begin{tabular}[c]{@{}l@{}}\end{tabular} \\ 

    \hline
    \begin{tabular}[c]{@{}l@{}}HISKIN\_AI/\\nlp/models/Transformer\end{tabular} & \begin{tabular}[c]{@{}l@{}}Bert.ipynb\\Transformer\\(Greedy\_decoder)\\Transformer.ipynb\end{tabular} 
    & \begin{tabular}[c]{@{}l@{}}\end{tabular} \\ 
    \hline
    \end{tabular} 
\end{table}
    
    
    \item[4.]  Class Components: 
    \begin{itemize}
        \item[$\bullet$] Models: This is a folder that contains codes for all models used for word embedding, the multi-class sentiment analysis problem to classify texts. 
        \item[$\bullet$]Basic Embedding: This is a folder that contains word embedding models such as NNLM, Word2Vec, and FastText. These models convert words into vectors and apply activation functions to learn the semantic relationship between words.
        \item[$\bullet$]CNN: This is a folder that contains convolutional neural network(CNN)-based models suitable for text classification tasks.
        \item[$\bullet$]RNN:This is a folder that contains the  Recurrent Neural Network(RNN) models which can learn more complex sequence patterns with mitigation of gradient disappearance problems and introduction of gate mechanisms.
        \item[$\bullet$] Attention Mechanism: They are binary emotional classification models and translation models that use an attention mechanism. The attention mechanism allows us to observe which parts of the translation the model pays attention to, which helps us understand the model's internal behavior.
        \item[$\bullet$]Transformer: It is a Transformer model using PyTorch. Transformer is an architecture that performs sequence-to-sequence modeling using Attention Mechanism. After learning is completed, it outputs a translation of the test sentence and visualizes the attention weights in the last encoder and decoder layers, respectively.
        \item[$\bullet$]Chatbot\_transformer.ipynb: This is a file that contains key components of transforemers and chatbot models using transformers.
        \item[$\bullet$]Data Folder: This is a folder that contains 11,823 Question \& Answer datasets for chatbot training.
        \item[$\bullet$]OpenAI Model: This is a folder that implements a chat model using OpenAI API and a llm model using langchain, and deals with various methods in which OpenAI can be used in chatbots.

    \end{itemize}

    \item[5.] Where It's Taken From: AI is used to receive user utterance information received from the nugu speaker and output a response. In addition, this can provide a more appropriate response to the user based on the previous conversation history with the user.\\
    
    \item[6.] How / Why you used it: We used openai because we can customize the prompt so that we can ask the user the appropriate questions and answers in the desired field. In addition, even though it is the user's first question, it is possible to to induce an appropriate response to the user's request by using an conversation template.
\end{itemize}

\subsection{Module 4: AI Facial Analysis}
\begin{itemize}
    \item [1.] Purpose: HISKIN app allows users to photograph their faces and check their skin conditions through machine learning. Skin images are classified into three categories: acne, skin redness, and bags under eyes. The purpose of machine learning is to make the app more effective by delivering feedback on skin conditions every day. \\
    
    \item[2.] Functionality: Users score their own behavior on the Challenge tab. Take one picture of your face. AI trains the model during the given epochs, and tracks the training and verification loss and accuracy at each epoch. Whenever the optimal verification loss is updated, the model is saved so that the optimal model can be reused in the future. We visualize the training and verification loss and performance indicators (accuracy) of the model by epoch. The function takes the results of the model trained as input and draws the training and verification loss graph and the accuracy graph. We visualize the Confusion Matrix for the test data. The accuracy of image classification is about 77\%. Based on this, it has been confirmed that skin images can also be classified through machine learning. We can apply these results to our project to give app users feedback on their skin.
\\
    
    \item[3.] Location of Source Code: HYU-SE-HISKIN/HISKIN\_AI\_FacialAnalysis\\

    \begin{table} [h]
    \begin{tabular}{p{3cm}|p{2.5cm}|p{1.5cm}}
    \hline
    \begin{tabular}[c]{@{}l@{}}HISKIN\_AI\\\_FacialAnalysis\end{tabular} & \begin{tabular}[c]{@{}l@{}}archive\\main.ipynb\end{tabular}  & \begin{tabular}[c]{@{}l@{}}\end{tabular} \\ 

    \hline
    \begin{tabular}[c]{@{}l@{}}HISKIN\_AI\\\_FacialAnalysis/\\archive\end{tabular} & \begin{tabular}[c]{@{}l@{}}files\\skin\_defects.csv\end{tabular}  & \begin{tabular}[c]{@{}l@{}}\end{tabular} \\ 

     \hline
    \begin{tabular}[c]{@{}l@{}}HISKIN\_AI\\\_FacialAnalysis/\\archive/files\end{tabular} & \begin{tabular}[c]{@{}l@{}}acne\\bags\\redness\\\end{tabular}  & \begin{tabular}[c]{@{}l@{}}\end{tabular} \\ 
    \hline
    \end{tabular} 
\end{table}
    
    
    \item[4.]  Class Components: 
    \begin{itemize}
        \item[$\bullet$]main.ipynb: It is a file that contains a model in which when a user takes a picture of his or her face, AI can classify it into three categories: acne, skin redness, and bags under eyes. 
        \item[$\bullet$]archive/skin\_defects.csv: It is a file that contains data frames classified into five labels for the image dataset.
        \item[$\bullet$]archive/files: It is a folder that contains a total of 30 front-facing, left- and right-facing image datasets for acne, skin redness, and bags under eyes.
    \end{itemize}

    \item[5.] Where It's Taken From: When a user takes a picture of his or her face, AI classifies it into three categories: acne, skin redness, and bags. Based on this, the application provides feedback on the skin condition to the user by combining the challenge score and the face recognition results.\\
    
    \item[6.] How / Why you used it: We used Jupyter Notebook and Pytorch libraries. Jupyter Notebook is an open source web application for interactive computing and data visualization. It is widely used primarily in data science and machine learning, and presents codes, texts, figures, and formulas in a single document. And Pytorch is an open source library for deep learning and machine learning, characterized in particular by the use of dynamic computational graphs.

\end{itemize}
\clearpage

\section{Use Cases}
\subsection{Start Page}
\begin{figure}[h]
    \centering
    \includegraphics[width=.9\columnwidth]{fig/시작페이지1110.JPG}
    \label{fig:Start Page}
    \caption{Start Page} 
    \end{figure}
This page appears when the user opens the app for the first time. It provides a basic introduction to HI-SKIN, offering an initial explanation for users who are using the app for the first time.

\subsection{Log In}
 \begin{figure}[h]
    \centering
    \includegraphics[width=.43\columnwidth]{fig/HISKIN_231130-04.jpg}
    \label{fig:Log In Page}
    \caption{Log In Page} 
    \end{figure}
Log In Page is a page which allows users to log in by entering ID and Password. \\ \\ \\ \\ \\ \\

\subsection{Sign Up}
\begin{figure}[h]
    \centering
    \includegraphics[width=.43\columnwidth]{fig/HISKIN_231130-05.jpg}
    \label{fig:Sign Up Page}
    \caption{Sign Up Page} 
    \end{figure}
Users create an account by entering their name, ID, password, and gender information. If the user clicks the 'Sign Up' button after entering all the required information, they can proceed to the login screen and attempt to log in with the newly created account. \\

\subsection{Device Registration}
\begin{figure}[h]
    \centering
    \includegraphics[width=.9\columnwidth]{fig/디바이스등록1130.JPG}
    \label{fig:Device Registration Page}
    \caption{Device Registration Page} 
    \end{figure}
Users can register their beauty device on the dedicated page. An image on the initial screen prompts users to keep their smartphones close to the device. When the smartphone is brought near, the registration process begins, and the screen shows 'loading' during registration. Upon completion, a 'registration complete' message appears, transitioning to the 'My Device' screen. Here, users can view details of their registered device and return to the home screen with the 'Complete' button at the bottom. \\

\subsection{Home Page}
\begin{figure}[h]
    \centering
    \includegraphics[width=.4\columnwidth]{fig/HISKIN_231130-09.jpg}
    \label{fig:Home Page}
    \caption{Home Page} 
    \end{figure}
On the home page, users can access features such as device registration, viewing registered device information, skin type testing, challenges, cosmetics recommendations, and today's popular posts. By clicking on buttons like "Register New Device", "My Skin MBTI", and "Record Today's Skin Score", users can navigate to pages for device registration, skin type testing, and the challenge section, respectively.\\ 

\subsection{Skin Type Test}
\begin{figure}[h]
    \centering
    \includegraphics[width=.39\columnwidth]{fig/HISKIN_231130-10.jpg}
    \label{fig:Questions and Answers}
    \caption{Questions and Answers Page} 
    \end{figure}
On the skin type test screen, users respond to a total of 8 questions. The skin type questions are based on the skin type classification method by Dr. Leslie Baumann, a dermatologist in the United States, tailored to the characteristics of East Asian skin. Each question has a selection of 4 to 5 answers, and users choose one before moving on to the next set of questions by pressing the "Next" button.\\ 
\begin{figure}[h]
    \centering
    \includegraphics[width=.37\columnwidth]{fig/HISKIN_231130-12.jpg}
    \label{fig:Skin Type Result}
    \caption{Skin Type Result} 
    \end{figure} \\
Users can view the results of their skin type test on the skin type result screen. The skin type result is generated by analyzing the test scores among a total of 16 skin types. Users can learn about the characteristics of their skin type and also access information about the features of cosmetics suitable for their skin type. By clicking the "Finish Diagnosis" button, users can navigate back to the home page.

\subsection{Cosmetic Recommendations}
\begin{figure}[h]
    \centering
    \includegraphics[width=.36\columnwidth]{fig/HISKIN_231130-13.jpg}
    \label{fig:Cosmetic Recommendations}
    \caption{Cosmetic Recommendations} 
    \end{figure}
On the cosmetics recommendation screen, users receive information about their skin type and learn about the characteristics of cosmetics suitable for their skin type. The app recommends the top-ranking product based on popularity from the "Olive Young" site by searching for keywords that match the user's skin type.

\subsection{Skin Care Challenge}
\begin{figure}[h]
    \centering
    \includegraphics[width=\columnwidth]{fig/챌린지1111.JPG}
    \label{fig:Skin Care Challenge}
    \caption{Skin Care Challenge} 
    \end{figure}
On the challenge screen, users can respond to four challenges. The challenge questions consist of four items selected by the Korean Dermatological Association, focusing on important aspects of daily skincare. The response choices range from "Very Poor" to "Very Well," with a total of five options. Scores are assigned on a scale of 5, 10, 15, 20, and 25 points, resulting in a minimum challenge score of 20 and a maximum of 100.

\subsection{Care}
\begin{figure}[h]
    \centering
    \includegraphics[width=.36\columnwidth]{fig/HISKIN_231130-18.jpg}
    \label{fig:Care 1}
    \caption{Care 1} 
    \end{figure}
On the care screen, users can review the devices they have connected through device registration. Users have the option to choose the device they want to use for their skin care.\\  \\ \\ \\
\begin{figure}[h]
    \centering
    \includegraphics[width=.4\columnwidth]{fig/HISKIN_231130-19.jpg}
    \label{fig:Care 2}
    \caption{Care 2} 
    \end{figure}
\\The image above depicts the screen users see when selecting the device for use. Users can choose between two types of routines and select a skincare routine based on the theme they desire.
\begin{figure}[h]
    \centering
    \includegraphics[width=.4\columnwidth]{fig/HISKIN_231130-20.jpg}
    \label{fig:Care 2}
    \caption{Care 2} 
    \end{figure}
\\The image above represents the screen users encounter when selecting a routine. On the screen, there is a prompt instructing users to operate the device according to the speaker's voice guidance. Below the prompt, an image of the speaker output and the remaining time are displayed for users to check. \\ \\ \\

\subsection{Community} 
\begin{figure}[htb!]
    \centering
    \includegraphics[width=.42\columnwidth]{fig/HISKIN_231120-19.jpg}
    \label{fig:Community Page}
    \caption{Community Page} 
    \end{figure}
HI-SKIN features a community menu in the bottom tab where users can access comprehensive articles containing valuable information on device usage, skincare secrets, and more. Clicking on an article displays the information at the bottom of the app. Use the 'To List' button to return to the community screen. Pressing the heart button next to an article saves it as an 'attention article' for separate collection. \\

\subsection{My Page}
\begin{figure}[h]
    \centering
    \includegraphics[width=.37\columnwidth]{fig/HISKIN_231130-22.jpg}
    \label{fig:My Page 1}
    \caption{My Page 1} 
    \end{figure}
On the My Page screen, there are options for account settings, skin type retest, and skin reports. In the 'Account Settings' tab, users can modify the information they entered during registration. By clicking the 'Skin Type Retest' button, users can navigate to the skin type test screen to take the skin type test again. Pressing the 'Skin Report' button allows users to check their challenge score reports.\\
\begin{figure}[h]
    \centering
    \includegraphics[width=.4\columnwidth]{fig/1주일1130.JPG}
    \label{fig:My Page 2}
    \caption{My Page 2} 
    \end{figure}
\\The image above shows the screen users encounter when they click on the Skin Report. On this screen, users can view a graph representing their challenge scores for the past 7 days. Below the graph, the days of the week and corresponding challenge scores are displayed for users to review.\\
\begin{figure}[h]
    \centering
    \includegraphics[width=.4\columnwidth]{fig/3개월11111.jpg}
    \label{fig:My Page 3}
    \caption{My Page 3} 
    \end{figure}
\\The image above depicts the skin report for the last 3 months. It displays the average challenge scores over a period of 15 days in a graph format. Below the graph, the dates and corresponding challenge scores are shown for users to review.\\
\begin{figure}[h]
    \centering
    \includegraphics[width=.4\columnwidth]{fig/6개월11111.jpg}
    \label{fig:My Page 4}
    \caption{My Page 4} 
    \end{figure}
\\The image above represents the skin report for the last 6 months. It illustrates the average challenge scores over one month in a graph format. Below the graph, the months and corresponding challenge scores are displayed for users to review.



\bibliographystyle{IEEEtran} 
\bibliography{references}


\end{document}
